\documentclass[./main.tex]{subfiles}

\begin{document}
\chapter*{the process of human learning}

The word `humanly' is important, as mathematics, science, philosophy etc... are after all, human made things and there is nothing so `absolute' about it.
The word `humanly' inherently makes the process of anything related to that (even the mighty mathematics) an iterative process.
In that nothing we conclude, discover, invent ever can be trusted a full 100\% and should constantly be challenged.

It is in the balance between having trust and challenging it simultaneously lies any `conceivable progress'.
If the balance tips left it is called `arrogance', else if it tips right it is called  `insanity'.
If the balance is right I feel that it produces a sense of happiness, fun and progress.

I believe that this feeling is what human beings pursuing science crave for (at least me).
Also I personally have a application-oriented mindset, as in if I can apply my knowledge to create/improve a tangible system, it produces a great sense of joy to me.
So I guess if I bring some kind of application to every piece of new thing I learn then I might have a deeper (a more intimate) insight in it.

\paragraph{To summarize}
\begin{enumerate}
  \item It is not only good but also important to challenge what you trust, but with control.
  \item Periodically, it is very healthy to think that whatever you learned until now is a complete non-sense and reset your brain fully.
  \item Draw an application to everything you learn, you will learn better this way.
  \item The whole point of pursuing math and science it to pursue the joy in it. So better stop feeling so seriously about it. \textbf{Never forget this}.
\end{enumerate}

\end{document}

