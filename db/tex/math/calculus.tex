\documentclass[./main.tex]{subfiles}

\begin{document}
\chapter{infinitesimal calculus}
\todo
\begin{enumerate}[nolistsep]
  \item limits
  \item the beauty of the phrase exact approximation
  \item partial derivatives
  \item derivative of vectors and matrices, jacobians, hessians, laplacians etc...
\end{enumerate}


\addcontentsline{toc}{section}{Derivative}
\begin{statement}[\textbf{Derivative}]
\label{statement:Derivative}\hspace*{0pt}\par
\end{statement}
\textbf{Description}:
The derivative of a [\hyperref[statement:Function]{function}] $\Lambda : D \to C $ at $\alpha \in D$ is the limiting value of
\[
  \frac{\Lambda(\alpha + h) - \Lambda(\alpha)}{h}
\]
as h tends to zero.
Denoted by
\[
  \frac{d\Lambda}{d\alpha} \equiv \lim_{h \to 0}\frac{\Lambda(\alpha + h) - \Lambda(\alpha)}{h}
\]
\par
{\color{magenta} \textbf{Significance}:
A notion of difference in function given difference in input.
Generally visualized as a slope of the curve defined by the function.
It is a general misconception that derivative is something very complicated and uncomprehensible thing.
But it is actually not.
It is a simple thing.

One subtle but important thing to notice here is that, there is no usage of the term "infinitesimals" here.
The spirit of the derivative is in measuring change in function when input is changed.
And the actual derivative itself is the value that the ratio approaches as the change gets smaller and smaller.

It can definitely be the case that the ratio does not have a limiting value.
The derivative is not defined for such cases, that's all.
\par}
\begin{proof}
\proofbydefinition
\end{proof}\par
\paragraph{1 parents} \hyperref[statement:Function]{Function}, 
\paragraph{1 children} \hyperref[statement:Fundamental theorem of Calculus]{Fundamental theorem of Calculus}, 



\addcontentsline{toc}{section}{Integral}
\begin{statement}[\textbf{Integral}]
\label{statement:Integral}\hspace*{0pt}\par
\end{statement}
\textbf{Description}:
The integral of a [\hyperref[statement:Function]{function}] $\Lambda : D \to C $ at $\alpha \in D$ is the limiting value of
\[
  \sum_{i=0}^{n-1} \Lambda(\alpha * \frac{i}{n}) * \frac{\alpha}{n}
\]
as n tends to infinity.
Denoted by
\[
  \int_{0}^{\alpha} \Lambda(\beta) d\beta \equiv \lim_{n \to \infty}\sum_{i=0}^{n-1} \Lambda(\alpha * \frac{i}{n}) * \frac{\alpha}{n}
\]
\par
{\color{magenta} \textbf{Significance}:
A notion of sum of function until a given input.
Generally visualized as area under the curve defined by the function.
It is a general misconception that integral is some other worldly incomprehensible thing.
It is not.
It is just this.
There is nothing more to the definition.

One thing to observe is that there is no method to natively evaluate the integral contrary to derivatives (which uses limits).
\par}
\begin{proof}
\proofbydefinition
\end{proof}\par
\paragraph{1 parents} \hyperref[statement:Function]{Function}, 
\paragraph{1 children} \hyperref[statement:Fundamental theorem of Calculus]{Fundamental theorem of Calculus}, 



\addcontentsline{toc}{section}{Fundamental theorem of Calculus}
\begin{statement}[\textbf{Fundamental theorem of Calculus}]
\label{statement:Fundamental theorem of Calculus}\hspace*{0pt}\par
\end{statement}
\textbf{Description}:
This theorem links [\hyperref[statement:Derivative]{derivative}]s and [\hyperref[statement:Integral]{integral}]s.

For [\hyperref[statement:Function]{function}]s $g, f:D \to C$ s.t. $\forall \alpha \in D$,
\[
  f(\alpha) \equiv \int_{0}^{\alpha} g(\beta) d\beta
  \impl
  \frac{df(\alpha)}{d\alpha} \equiv g(\alpha)
\]
\[
  \frac{df(\alpha)}{d\alpha} \equiv g(\alpha)
  \impl
  f(\alpha) \equiv f(0) + \int_{0}^{\alpha} g(\beta) d\beta
\]
.
\par
{\color{magenta} \textbf{Significance}:
Also provides a method to evaluate integrals.
There is no native method to evaluate integral other than to basically guess viz. guess a function which upon derivation gives the original function and get its value at required point.

It is almost hilarious.
It almost seems like the integrals are created like a hack.
This should be developed and we should make a way to natively evaluate integrals that does not involve guessing.
\par}
\begin{proof}
The term `approximation' used here is more profound than you might think.
\\
\textbf{First}
\\
If $f$ is the integral function i.e. the area under the curve, consider a small finite change in the area $df$.
Then
\[
  df \approx g(\alpha) * d\alpha
\]
\[
  \frac{df}{d\alpha} \approx g(\alpha)
\]
As the $d\alpha$ tends to 0
\begin{enumerate}[nolistsep]
  \item The limiting value of the LHS is derivative of $f$
  \item The limiting value of the RHS is $g(\alpha)$
  \item The approximation becomes equivalence
\end{enumerate}
Therefore,
\[
  \frac{df}{d\alpha} \equiv g(\alpha)
\]
\textbf{Second}
\\
If $g$ is the derivative of $f$
\[
  g(\beta) * d\beta \approx f(\beta + d\beta) - f(\beta)
\]
\[
  \sum_{0}^{n-1} g(\alpha * \frac{i}{n}) * \frac{\alpha}{n} \approx \sum_{0}^{n-1} f(\alpha * \frac{i + 1}{n}) - f(\alpha * \frac{i}{n})
\]
Expanding and cancelling terms on RHS we have
\[
  \sum_{0}^{n-1} g(\alpha * \frac{i}{n}) * \frac{\alpha}{n} \approx f(\alpha) - f(0)
\]
As $n$ tends to infinity
\begin{enumerate}[nolistsep]
  \item The limiting value of the LHS is derivative of $\int_{0}^{\alpha} g(\beta) d\beta$
  \item The limiting value of the RHS is $f(\alpha) - f(0)$
  \item The approximation becomes equivalence
\end{enumerate}
Therefore,
\[
  f(\alpha) - f(0) \equiv \int_{0}^{\alpha} g(\beta) d\beta
\]
\[
  f(\alpha) \equiv f(0) + \int_{0}^{\alpha} g(\beta) d\beta
\]
\end{proof}\par
\paragraph{3 parents} \hyperref[statement:Derivative]{Derivative}, \hyperref[statement:Integral]{Integral}, \hyperref[statement:Function]{Function}, 
\paragraph{0 children} 


\end{document}


