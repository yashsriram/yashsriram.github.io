\documentclass[../main.tex]{subfiles}

\begin{document}
\chapter{coordinate system}


\addcontentsline{toc}{section}{Point}
\begin{statement}[\textbf{Point}]
\label{statement:Point}\hspace*{0pt}\par
\end{statement}
\textbf{Description}:
A point represents state of an object.
A point does not change, so there is no notion of moving a point.
Objects can change states and hence can go from one point to another.
\par
{\color{magenta} \textbf{Significance}:
A basic tool for representing state of objects.
For instance a point can represent position, temperature, velocity, pressure etc...
\par}
\begin{proof}Axiom.\end{proof}\par
\paragraph{0 parents} 
\paragraph{6 children} \hyperref[statement:Space]{Space}, \hyperref[statement:State Vector]{State Vector}, \hyperref[statement:Coordinate System]{Coordinate System}, \hyperref[statement:Coordinate Axis]{Coordinate Axis}, \hyperref[statement:Straight Space]{Straight Space}, \hyperref[statement:Position Space]{Position Space}, 



\addcontentsline{toc}{section}{Space}
\begin{statement}[\textbf{Space}]
\label{statement:Space}\hspace*{0pt}\par
\end{statement}
\textbf{Description}:
A space is a [\hyperref[statement:Set]{set}] of [\hyperref[statement:Point]{point}]s.
\par
{\color{magenta} \textbf{Significance}:
A tool to group all states considered in the context.
As point does not change, space does not change.
\par}
\begin{proof}
\proofbydefinition
\end{proof}\par
\paragraph{2 parents} \hyperref[statement:Set]{Set}, \hyperref[statement:Point]{Point}, 
\paragraph{7 children} \hyperref[statement:Coordinate System]{Coordinate System}, \hyperref[statement:Coordinate Axis]{Coordinate Axis}, \hyperref[statement:Straight Space]{Straight Space}, \hyperref[statement:Position Space]{Position Space}, \hyperref[statement:Linear Transformation]{Linear Transformation}, \hyperref[statement:Rotation Transformation]{Rotation Transformation}, \hyperref[statement:Definiteness]{Definiteness}, 



\addcontentsline{toc}{section}{State Vector}
\begin{statement}[\textbf{State Vector}]
\label{statement:State Vector}\hspace*{0pt}\par
\end{statement}
\textbf{Description}:
A state vector is a [\hyperref[statement:Vector]{vector}] that represents a [\hyperref[statement:Point]{point}].
\par
{\color{magenta} \textbf{Significance}:
This is not a redundant definition (with point).
Although a point never changes, the representation of that point can change.
That is why we define both point and state vector.
\par}
\begin{proof}
\proofbydefinition
\end{proof}\par
\paragraph{2 parents} \hyperref[statement:Vector]{Vector}, \hyperref[statement:Point]{Point}, 
\paragraph{3 children} \hyperref[statement:State Vector Space]{State Vector Space}, \hyperref[statement:Coordinate Axis]{Coordinate Axis}, \hyperref[statement:Straight Space]{Straight Space}, 



\addcontentsline{toc}{section}{State Vector Space}
\begin{statement}[\textbf{State Vector Space}]
\label{statement:State Vector Space}\hspace*{0pt}\par
\end{statement}
\textbf{Description}:
A state vector space is a [\hyperref[statement:Set]{set}] of all possible N-dimensional [\hyperref[statement:State Vector]{state vector}]s.
\par
{\color{magenta} \textbf{Significance}:
Used to represent a collection of state vectors.
Mainly created for constructing coordinate system.
\par}
\begin{proof}
\proofbydefinition
\end{proof}\par
\paragraph{2 parents} \hyperref[statement:Set]{Set}, \hyperref[statement:State Vector]{State Vector}, 
\paragraph{3 children} \hyperref[statement:Coordinate System]{Coordinate System}, \hyperref[statement:Linear Transformation]{Linear Transformation}, \hyperref[statement:Rotation Transformation]{Rotation Transformation}, 



\addcontentsline{toc}{section}{Coordinate System}
\begin{statement}[\textbf{Coordinate System}]
\label{statement:Coordinate System}\hspace*{0pt}\par
\end{statement}
\textbf{Description}:
A coordinate system $ \Lambda $ is a [\hyperref[statement:Bijection]{bijection}] mapping from [\hyperref[statement:Space]{space}] G to an N-dimensional [\hyperref[statement:State Vector Space]{state vector space}] V.

Denoted by $ \Lambda: G \to V $. $\forall g \in G, \Lambda(g) $ (i.e. the image of [\hyperref[statement:Point]{point}] g under coordinate system $ \Lambda $) can be denoted by $P^{\Lambda}_{g}$.
\par
{\color{magenta} \textbf{Significance}:
We choose state vector space to represent a space G because, by definition any v $\in V \equiv $ linear combination of all vectors of standard basis.
Therefore we only need standard basis to describe any vector in V, nothing else.

Sets up a representation for every point in space uniquely i.e. such that any point's representation is different from all others.
Defining it like this removes all untagible invisible evil chains that are otherwise tied to the concept, thus providing a clean, elegant and concrete description.

Here we make sort of a first proper connection from points and spaces to vectors.
Note that when we talk about dimensionality here, it is the property of vectors not the space itself.
Thus space and dimensionality are decoupled, this means a space can be represented by different dimensional vectors.
\par}
\begin{proof}
\proofbydefinition
\end{proof}\par
\paragraph{4 parents} \hyperref[statement:Bijection]{Bijection}, \hyperref[statement:Space]{Space}, \hyperref[statement:State Vector Space]{State Vector Space}, \hyperref[statement:Point]{Point}, 
\paragraph{4 children} \hyperref[statement:Coordinate Axis]{Coordinate Axis}, \hyperref[statement:Straight Space]{Straight Space}, \hyperref[statement:Linear Transformation]{Linear Transformation}, \hyperref[statement:Rotation Transformation]{Rotation Transformation}, 



\addcontentsline{toc}{section}{Coordinate Axis}
\begin{statement}[\textbf{Coordinate Axis}]
\label{statement:Coordinate Axis}\hspace*{0pt}\par
\end{statement}
\textbf{Description}:
A coordinate axis of a [\hyperref[statement:Coordinate System]{coordinate system}] $ \Lambda $ acting on a [\hyperref[statement:Space]{space}] G is a [\hyperref[statement:Subset]{subset}] of G, say A such that the [\hyperref[statement:State Vector]{state vector}] of any [\hyperref[statement:Point]{point}] in A $ \equiv $ a scaled version of an [\hyperref[statement:Standard Vector]{standard vector}] B [\hyperref[statement:Vector Scaling]{vector scaling}].
\par
{\color{magenta} \textbf{Significance}:
For a coordinate system mapping to an N-dimensional state vector set, there are N coordinate axes.
Due to the virtue of state vectors on coordinate axis being scaled standard vectors, any linear combination of two vectors on a coordinate axis lies on the same coordinate axis.
It has no connection with other coordinate axes.
This is a good thing because it makes things simple, decoupled and non-clumsy.

A coordinate axis is a simple way to disassemble a coordinate system.
Any point in the state vector space $ \equiv $ linear combination of some point on each coordinate axis.

Misconception busting:
\begin{enumerate}[nolistsep]
  \item There is no need to name the axes X, Y ...
  \item There is no need for the axes to be perpendicular.
  \item No need for 1 unit along first axis to be same as 1 unit along second axis.
  \item No need for units of first axis to be same as the second one
\end{enumerate}
\par}
\begin{proof}
\proofbydefinition
\end{proof}\par
\paragraph{7 parents} \hyperref[statement:Coordinate System]{Coordinate System}, \hyperref[statement:Space]{Space}, \hyperref[statement:Subset]{Subset}, \hyperref[statement:State Vector]{State Vector}, \hyperref[statement:Point]{Point}, \hyperref[statement:Standard Vector]{Standard Vector}, \hyperref[statement:Vector Scaling]{Vector Scaling}, 
\paragraph{0 children} 



\addcontentsline{toc}{section}{Straight Space}
\begin{statement}[\textbf{Straight Space}]
\label{statement:Straight Space}\hspace*{0pt}\par
\end{statement}
\textbf{Description}:
A straight space under a [\hyperref[statement:Coordinate System]{coordinate system}] $\Lambda$ acting on a [\hyperref[statement:Space]{space}] G is a [\hyperref[statement:Subset]{subset}] of G, say S such that the [\hyperref[statement:State Vector]{state vector}] of any [\hyperref[statement:Point]{point}] in S $ \equiv $ $ o + t * d $ where $ o, d \in G $ and t $ \in $ [\hyperref[statement:Real Numbers]{real numbers}].
\par
{\color{magenta} \textbf{Significance}:
If the space is finite, it is called a line segement otherwise a line.
We can see the importance of the operation of linear combination here, as it is the defining operation for lines and line segments.
Generally denoted by drawing $\overleftrightarrow{}$ on paper.
\par}
\begin{proof}
\proofbydefinition
\end{proof}\par
\paragraph{6 parents} \hyperref[statement:Coordinate System]{Coordinate System}, \hyperref[statement:Space]{Space}, \hyperref[statement:Subset]{Subset}, \hyperref[statement:State Vector]{State Vector}, \hyperref[statement:Point]{Point}, \hyperref[statement:Real Numbers]{Real Numbers}, 
\paragraph{0 children} 



\addcontentsline{toc}{section}{Position Space}
\begin{statement}[\textbf{Position Space}]
\label{statement:Position Space}\hspace*{0pt}\par
\end{statement}
\textbf{Description}:
A [\hyperref[statement:Space]{space}] where each [\hyperref[statement:Point]{point}] represents a position of an object is called a position space.
The coordinate system that acts on it is called position coordinate system.
\par
{\color{magenta} \textbf{Significance}:
Used to denote positions of objects in the real world.
Positions in real world are generally mapped to a 3D vector set.
This mapping is just based on the belief that the real world is in 3D.
But as we believe, beliefs are not final and should be challanged constantly and consciously.

For example positions can be mapped to 4D vector set, the new dimension being the time.
Similarly more dimensions can represent more information.
Or the real world might contain something more that we have not yet observed properly which cannot be represented using a 3D vector set.

Whatever the real world might be, we can always imagine a space represented by 3D vector set and do operations on it.

\par}
\begin{proof}
\proofbydefinition
\end{proof}\par
\paragraph{2 parents} \hyperref[statement:Space]{Space}, \hyperref[statement:Point]{Point}, 
\paragraph{0 children} 


\end{document}
