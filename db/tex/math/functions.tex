\documentclass[../main.tex]{subfiles}

\begin{document}
\chapter{functions}


\addcontentsline{toc}{section}{Function}
\begin{statement}[\textbf{Function}]
\label{statement:Function}\hspace*{0pt}\par
\end{statement}
\textbf{Description}:
A function $ \Lambda $ on a [\hyperref[statement:Set]{set}] called its domain $ D $ is a mapping from each element $ \alpha $ that [\hyperref[statement:Belongs To]{belongs to}] $ D $ to an element $ \beta $ in a new set C called its co-domain.
This mapping exists for all elements in the domain and each element in domain is mapped to only one element in co-domain.
For each element $ \alpha \in D $ the mapping $ \beta $ is called image of $ \alpha $.
The set of images for all elements the domain is called range $ R $ of the function $ \Lambda $.
Function is denoted by $ \Lambda:D \to C $. The image of $ \alpha $ is denoted by $ \Lambda(\alpha) \forall \alpha \in D $.
\par
{\color{magenta} \textbf{Significance}:
A basic tool for moving b/w sets.
Note that two elements on domain can be mapped to same element in co-domain.
Also range $ \subseteq $ co-domain.
\par}
\begin{proof}
\proofbydefinition
\end{proof}\par
\paragraph{2 parents} \hyperref[statement:Set]{Set}, \hyperref[statement:Belongs To]{Belongs To}, 
\paragraph{4 children} \hyperref[statement:Derivative]{Derivative}, \hyperref[statement:Integral]{Integral}, \hyperref[statement:Fundamental theorem of Calculus]{Fundamental theorem of Calculus}, \hyperref[statement:Random Variable]{Random Variable}, 



\addcontentsline{toc}{section}{Injection}
\begin{statement}[\textbf{Injection}]
\label{statement:Injection}\hspace*{0pt}\par
\end{statement}
\textbf{Description}:{\color{red} \todo}\par
{\color{magenta} \textbf{Significance}:{\color{red} \todo}\par}
\begin{proof}Axiom.\end{proof}\par
\paragraph{0 parents} 
\paragraph{0 children} 



\addcontentsline{toc}{section}{Surjection}
\begin{statement}[\textbf{Surjection}]
\label{statement:Surjection}\hspace*{0pt}\par
\end{statement}
\textbf{Description}:{\color{red} \todo}\par
{\color{magenta} \textbf{Significance}:{\color{red} \todo}\par}
\begin{proof}Axiom.\end{proof}\par
\paragraph{0 parents} 
\paragraph{0 children} 



\addcontentsline{toc}{section}{Bijection}
\begin{statement}[\textbf{Bijection}]
\label{statement:Bijection}\hspace*{0pt}\par
\end{statement}
\textbf{Description}:{\color{red} \todo}\par
{\color{magenta} \textbf{Significance}:{\color{red} \todo}\par}
\begin{proof}Axiom.\end{proof}\par
\paragraph{0 parents} 
\paragraph{2 children} \hyperref[statement:Coordinate System]{Coordinate System}, \hyperref[statement:Linear Transformation]{Linear Transformation}, 


\end{document}
