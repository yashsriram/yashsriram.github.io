\documentclass[../main.tex]{subfiles}

\begin{document}
\chapter{numbers}



\addcontentsline{toc}{section}{Natural Numbers}
\begin{statement}[\textbf{Natural Numbers}]
\label{statement:Natural Numbers}\hspace*{0pt}\par
\end{statement}
\textbf{Description}:
  The [\hyperref[statement:Ordered Set]{ordered set}] of all numbers [\hyperref[statement:Number]{number}] used for counting and ordering is called the set of natural numbers. It is denoted by $ \mathbb{N}$.
\par
{\color{magenta} \textbf{Significance}:
  Encompasses all possible counts.
\par}
\begin{proof}
\proofbydefinition
\end{proof}\par
\paragraph{2 parents} \hyperref[statement:Ordered Set]{Ordered Set}, \hyperref[statement:Number]{Number}, 
\paragraph{2 children} \hyperref[statement:Whole Numbers]{Whole Numbers}, \hyperref[statement:Deterministic Experiment]{Deterministic Experiment}, 



\addcontentsline{toc}{section}{Whole Numbers}
\begin{statement}[\textbf{Whole Numbers}]
\label{statement:Whole Numbers}\hspace*{0pt}\par
\end{statement}
\textbf{Description}:
  The [\hyperref[statement:Union]{union}] of [\hyperref[statement:Natural Numbers]{natural numbers}] and [\hyperref[statement:Set]{set}] which contains only [\hyperref[statement:Zero]{zero}] . It is denoted by $ \mathbb{W}$.
\par
{\color{magenta} \textbf{Significance}:
  The name whole is given because we are gonna define numbers that will represent non-whole entities.
\par}
\begin{proof}
\proofbydefinition
\end{proof}\par
\paragraph{4 parents} \hyperref[statement:Union]{Union}, \hyperref[statement:Natural Numbers]{Natural Numbers}, \hyperref[statement:Set]{Set}, \hyperref[statement:Zero]{Zero}, 
\paragraph{0 children} 




\addcontentsline{toc}{section}{Additive Inverse}
\begin{statement}[\textbf{Additive Inverse}]
\label{statement:Additive Inverse}\hspace*{0pt}\par
\end{statement}
\textbf{Description}:{\color{red} \todo}\par
{\color{magenta} \textbf{Significance}:{\color{red} \todo}\par}
\begin{proof}Axiom.\end{proof}\par
\paragraph{0 parents} 
\paragraph{0 children} 




\addcontentsline{toc}{section}{Integers}
\begin{statement}[\textbf{Integers}]
\label{statement:Integers}\hspace*{0pt}\par
\end{statement}
\textbf{Description}:{\color{red} \todo}\par
{\color{magenta} \textbf{Significance}:{\color{red} \todo}\par}
\begin{proof}Axiom.\end{proof}\par
\paragraph{0 parents} 
\paragraph{0 children} 



\addcontentsline{toc}{section}{Multiplicative Inverse}
\begin{statement}[\textbf{Multiplicative Inverse}]
\label{statement:Multiplicative Inverse}\hspace*{0pt}\par
\end{statement}
\textbf{Description}:{\color{red} \todo}\par
{\color{magenta} \textbf{Significance}:{\color{red} \todo}\par}
\begin{proof}Axiom.\end{proof}\par
\paragraph{0 parents} 
\paragraph{0 children} 



\addcontentsline{toc}{section}{Rational Numbers}
\begin{statement}[\textbf{Rational Numbers}]
\label{statement:Rational Numbers}\hspace*{0pt}\par
\end{statement}
\textbf{Description}:{\color{red} \todo}\par
{\color{magenta} \textbf{Significance}:{\color{red} \todo}\par}
\begin{proof}Axiom.\end{proof}\par
\paragraph{0 parents} 
\paragraph{0 children} 




\addcontentsline{toc}{section}{Irrational Numbers}
\begin{statement}[\textbf{Irrational Numbers}]
\label{statement:Irrational Numbers}\hspace*{0pt}\par
\end{statement}
\textbf{Description}:{\color{red} \todo}\par
{\color{magenta} \textbf{Significance}:{\color{red} \todo}\par}
\begin{proof}Axiom.\end{proof}\par
\paragraph{0 parents} 
\paragraph{0 children} 




\addcontentsline{toc}{section}{Real Numbers}
\begin{statement}[\textbf{Real Numbers}]
\label{statement:Real Numbers}\hspace*{0pt}\par
\end{statement}
\textbf{Description}:{\color{red} \todo}\par
{\color{magenta} \textbf{Significance}:{\color{red} \todo}\par}
\begin{proof}Axiom.\end{proof}\par
\paragraph{0 parents} 
\paragraph{5 children} \hyperref[statement:Vector]{Vector}, \hyperref[statement:Vector Scaling]{Vector Scaling}, \hyperref[statement:Linear Combination]{Linear Combination}, \hyperref[statement:Dot Product]{Dot Product}, \hyperref[statement:Straight Space]{Straight Space}, 


\end{document}
