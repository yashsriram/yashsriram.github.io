\documentclass[../main.tex]{subfiles}

\begin{document}
\chapter{sets}


\addcontentsline{toc}{section}{Set}
\begin{statement}[\textbf{Set}]
\label{statement:Set}\hspace*{0pt}\par
\end{statement}
\textbf{Description}:
A set $S$ is a collection of elements from which any two elements picked without replacement are non-identical [\hyperref[statement:Identity]{identity}]. If the elements are denoted by $ e_1, e_2, e_3, ... e_n $ the set is denoted by $ S \equiv \{ e_1, e_2, e_3, ... e_n \} $ [\hyperref[statement:Enumerated Notation]{enumerated notation}].
\par
{\color{magenta} \textbf{Significance}:
To represent collection of differently colored balls in an urn, etc...
\par}
\begin{proof}
\proofbydefinition
\end{proof}\par
\paragraph{2 parents} \hyperref[statement:Identity]{Identity}, \hyperref[statement:Enumerated Notation]{Enumerated Notation}, 
\paragraph{23 children} \hyperref[statement:Belongs To]{Belongs To}, \hyperref[statement:Subset]{Subset}, \hyperref[statement:Superset]{Superset}, \hyperref[statement:Proper Subset]{Proper Subset}, \hyperref[statement:Proper Superset]{Proper Superset}, \hyperref[statement:Union]{Union}, \hyperref[statement:Intersection]{Intersection}, \hyperref[statement:Complement]{Complement}, \hyperref[statement:Universal Set]{Universal Set}, \hyperref[statement:Null Set]{Null Set}, \hyperref[statement:Ordered Set]{Ordered Set}, \hyperref[statement:Properties of sets]{Properties of sets}, \hyperref[statement:Partition]{Partition}, \hyperref[statement:Function]{Function}, \hyperref[statement:Whole Numbers]{Whole Numbers}, \hyperref[statement:Linearly Independent Set]{Linearly Independent Set}, \hyperref[statement:Span]{Span}, \hyperref[statement:Basis]{Basis}, \hyperref[statement:Standard basis]{Standard basis}, \hyperref[statement:Space]{Space}, \hyperref[statement:State Vector Space]{State Vector Space}, \hyperref[statement:Outcome Space]{Outcome Space}, \hyperref[statement:Event Space]{Event Space}, 



\addcontentsline{toc}{section}{Belongs To}
\begin{statement}[\textbf{Belongs To}]
\label{statement:Belongs To}\hspace*{0pt}\par
\end{statement}
\textbf{Description}:
An element $ \lambda $ belongs to a [\hyperref[statement:Set]{set}] S $ \biimpl $ $ \exists! \tau \in S \mid \lambda \equiv \tau $ [\hyperref[statement:Identity]{identity}].
Denoted by $ \lambda \in S $.

An element $ \lambda $ does not belong to a S $ \biimpl $ $ \nexists \tau \in S \mid \lambda \equiv \tau $.
Denoted by $ \lambda \notin S $.
\par
{\color{magenta} \textbf{Significance}:
To indicate whether an element belongs to a set or not.
There cannot be more than one element in S that is identical to $ \lambda $ anyways due to the definition of a set.
\par}
\begin{proof}
\proofbydefinition
\end{proof}\par
\paragraph{2 parents} \hyperref[statement:Set]{Set}, \hyperref[statement:Identity]{Identity}, 
\paragraph{3 children} \hyperref[statement:Subset]{Subset}, \hyperref[statement:Complement]{Complement}, \hyperref[statement:Function]{Function}, 



\addcontentsline{toc}{section}{Subset}
\begin{statement}[\textbf{Subset}]
\label{statement:Subset}\hspace*{0pt}\par
\end{statement}
\textbf{Description}:
A [\hyperref[statement:Set]{set}] E is a subset of set F $ \biimpl $ $ \forall e \in E \impl e \in F $. Denoted by $ E \subseteq F $ [\hyperref[statement:Belongs To]{belongs to}].
\par
{\color{magenta} \textbf{Significance}:
Can be used to represent useful pieces of a set.
\par}
\begin{proof}
\proofbydefinition
\end{proof}\par
\paragraph{2 parents} \hyperref[statement:Set]{Set}, \hyperref[statement:Belongs To]{Belongs To}, 
\paragraph{6 children} \hyperref[statement:Superset]{Superset}, \hyperref[statement:Proper Superset]{Proper Superset}, \hyperref[statement:Partition]{Partition}, \hyperref[statement:Coordinate Axis]{Coordinate Axis}, \hyperref[statement:Straight Space]{Straight Space}, \hyperref[statement:Event]{Event}, 



\addcontentsline{toc}{section}{Superset}
\begin{statement}[\textbf{Superset}]
\label{statement:Superset}\hspace*{0pt}\par
\end{statement}
\textbf{Description}:
A [\hyperref[statement:Set]{set}] E is a superset of set F $ \biimpl $ $ F \subseteq E $ [\hyperref[statement:Subset]{subset}]. Denoted by $ E \supseteq F $.
\par
{\color{magenta} \textbf{Significance}:
The inverse of subset.
\par}
\begin{proof}
\proofbydefinition
\end{proof}\par
\paragraph{2 parents} \hyperref[statement:Set]{Set}, \hyperref[statement:Subset]{Subset}, 
\paragraph{0 children} 



\addcontentsline{toc}{section}{Proper Subset}
\begin{statement}[\textbf{Proper Subset}]
\label{statement:Proper Subset}\hspace*{0pt}\par
\end{statement}
\textbf{Description}:
A [\hyperref[statement:Set]{set}] E is a proper subset of set F $ \biimpl $  $ E \subseteq F $ and $ E \not\equiv F $ [\hyperref[statement:Identity]{identity}]. Denoted by $ E \subset F $.
\par
{\color{magenta} \textbf{Significance}:
Can be used to represent useful pieces of a set.
\par}
\begin{proof}
\proofbydefinition
\end{proof}\par
\paragraph{2 parents} \hyperref[statement:Set]{Set}, \hyperref[statement:Identity]{Identity}, 
\paragraph{0 children} 



\addcontentsline{toc}{section}{Proper Superset}
\begin{statement}[\textbf{Proper Superset}]
\label{statement:Proper Superset}\hspace*{0pt}\par
\end{statement}
\textbf{Description}:
A [\hyperref[statement:Set]{set}] E is a superset of set F $ \biimpl $ $ F \subset E $ [\hyperref[statement:Subset]{subset}]. Denoted by $ E \supset F $.
\par
{\color{magenta} \textbf{Significance}:
The inverse of subset.
\par}
\begin{proof}
\proofbydefinition
\end{proof}\par
\paragraph{2 parents} \hyperref[statement:Set]{Set}, \hyperref[statement:Subset]{Subset}, 
\paragraph{0 children} 


\addcontentsline{toc}{section}{Union}
\begin{statement}[\textbf{Union}]
\label{statement:Union}\hspace*{0pt}\par
\end{statement}
\textbf{Description}:
A union of two [\hyperref[statement:Set]{set}]s A and B is a set which contains every element in A, every element in B, with repetitions removed, if any. Denoted by $ A \cup B $.
\par
{\color{magenta} \textbf{Significance}:
A basic operator to combine sets.
\par}
\begin{proof}
\proofbydefinition
\end{proof}\par
\paragraph{1 parents} \hyperref[statement:Set]{Set}, 
\paragraph{2 children} \hyperref[statement:Universal Set]{Universal Set}, \hyperref[statement:Whole Numbers]{Whole Numbers}, 



\addcontentsline{toc}{section}{Intersection}
\begin{statement}[\textbf{Intersection}]
\label{statement:Intersection}\hspace*{0pt}\par
\end{statement}
\textbf{Description}:
An intersection of two [\hyperref[statement:Set]{set}]s A and B is a set which contains every element that is common to A and B. Denoted by $ A \cap B $.
\par
{\color{magenta} \textbf{Significance}:
A basic operator to combine sets.
\par}
\begin{proof}
\proofbydefinition
\end{proof}\par
\paragraph{1 parents} \hyperref[statement:Set]{Set}, 
\paragraph{0 children} 



\addcontentsline{toc}{section}{Complement}
\begin{statement}[\textbf{Complement}]
\label{statement:Complement}\hspace*{0pt}\par
\end{statement}
\textbf{Description}:
A complement of a [\hyperref[statement:Set]{set}] A is a set which contains all elements which do not [\hyperref[statement:Belongs To]{belongs to}] the set A itself. Denoted by $ \overline{A} $
\par
{\color{magenta} \textbf{Significance}:
A basic unary operator on a set.
\par}
\begin{proof}
\proofbydefinition
\end{proof}\par
\paragraph{2 parents} \hyperref[statement:Set]{Set}, \hyperref[statement:Belongs To]{Belongs To}, 
\paragraph{1 children} \hyperref[statement:Universal Set]{Universal Set}, 



\addcontentsline{toc}{section}{Universal Set}
\begin{statement}[\textbf{Universal Set}]
\label{statement:Universal Set}\hspace*{0pt}\par
\end{statement}
\textbf{Description}:
A universal set of a [\hyperref[statement:Set]{set}] A is a set which is identical to $ A \cup \overline{A} $ [\hyperref[statement:Identity]{identity}] [\hyperref[statement:Union]{union}] [\hyperref[statement:Complement]{complement}].
\par
{\color{magenta} \textbf{Significance}:
A representation of a whole. This is highly context dependent i.e. it depends on what one believes to be in the complement.
A proper superset of universal set doesn't exist.
One can argue to just put some more elements to build a superset, but it makes things complicated.
If there is ever a real need for such a construct (dynamic universal set!!!???), we can build it then.
\par}
\begin{proof}
\proofbydefinition
\end{proof}\par
\paragraph{4 parents} \hyperref[statement:Set]{Set}, \hyperref[statement:Identity]{Identity}, \hyperref[statement:Union]{Union}, \hyperref[statement:Complement]{Complement}, 
\paragraph{2 children} \hyperref[statement:Properties of sets]{Properties of sets}, \hyperref[statement:Outcome Space]{Outcome Space}, 



\addcontentsline{toc}{section}{Null Set}
\begin{statement}[\textbf{Null Set}]
\label{statement:Null Set}\hspace*{0pt}\par
\end{statement}
\textbf{Description}:
A null set is a [\hyperref[statement:Set]{set}] which has [\hyperref[statement:Zero]{zero}] elements in it. Denoted by $ \phi $.
\par
{\color{magenta} \textbf{Significance}:
A representation for nothingness.
The reason for my take is because even null set is just a mathematical tool.
There is a great debate whether there is a single null set or many null sets.
My take on it is that it depends on the context.
\par}
\begin{proof}
\proofbydefinition
\end{proof}\par
\paragraph{2 parents} \hyperref[statement:Set]{Set}, \hyperref[statement:Zero]{Zero}, 
\paragraph{7 children} \hyperref[statement:Properties of sets]{Properties of sets}, \hyperref[statement:Partition]{Partition}, \hyperref[statement:Mutually Exclusive Events]{Mutually Exclusive Events}, \hyperref[statement:Properties of probability]{Properties of probability}, \hyperref[statement:Conditional Probability]{Conditional Probability}, \hyperref[statement:Total Probability Theorem]{Total Probability Theorem}, \hyperref[statement:Bayes Theorem]{Bayes Theorem}, 





\addcontentsline{toc}{section}{Ordered Set}
\begin{statement}[\textbf{Ordered Set}]
\label{statement:Ordered Set}\hspace*{0pt}\par
\end{statement}
\textbf{Description}:
An ordered set is a [\hyperref[statement:Set]{set}] where there exists an [\hyperref[statement:Order]{order}] b/w every pair of elements in it.
\par
{\color{magenta} \textbf{Significance}:
Useful to represent set of heights of students in a class, etc...
\par}
\begin{proof}
\proofbydefinition
\end{proof}\par
\paragraph{2 parents} \hyperref[statement:Set]{Set}, \hyperref[statement:Order]{Order}, 
\paragraph{3 children} \hyperref[statement:Continuous Set]{Continuous Set}, \hyperref[statement:Contiguous Set]{Contiguous Set}, \hyperref[statement:Natural Numbers]{Natural Numbers}, 



\addcontentsline{toc}{section}{Continuous Set}
\begin{statement}[\textbf{Continuous Set}]
\label{statement:Continuous Set}\hspace*{0pt}\par
\end{statement}
\textbf{Description}:
An [\hyperref[statement:Ordered Set]{ordered set}] which has no discrete separation b/w any two of its elements is a continuous set.
\par
{\color{magenta} \textbf{Significance}:
To represent sets like the set of all wavelengths in the visible spectrum of light
\par}
\begin{proof}
\proofbydefinition
\end{proof}\par
\paragraph{1 parents} \hyperref[statement:Ordered Set]{Ordered Set}, 
\paragraph{0 children} 



\addcontentsline{toc}{section}{Contiguous Set}
\begin{statement}[\textbf{Contiguous Set}]
\label{statement:Contiguous Set}\hspace*{0pt}\par
\end{statement}
\textbf{Description}:
An [\hyperref[statement:Ordered Set]{ordered set}] which has a discrete separation b/w any two of its elements is a contiguous set.
\par
{\color{magenta} \textbf{Significance}:
To represent colors in the rainbow in order [Violet, Indigo, Blue, Green, Yellow, Orange, Red]
\par}
\begin{proof}
\proofbydefinition
\end{proof}\par
\paragraph{1 parents} \hyperref[statement:Ordered Set]{Ordered Set}, 
\paragraph{0 children} 



\addcontentsline{toc}{section}{Properties of sets}
\begin{statement}[\textbf{Properties of sets}]
\label{statement:Properties of sets}\hspace*{0pt}\par
\end{statement}
\textbf{Description}:
The following list enumerates some properties of [\hyperref[statement:Set]{set}]s. Here $S$ denotes [\hyperref[statement:Universal Set]{universal set}] and $\phi$ denotes [\hyperref[statement:Null Set]{null set}] [\hyperref[statement:Identity]{identity}].
\begin{enumerate}[nolistsep]
  \item $ A \cup B \equiv B \cup A $ (Commutative)
  \item $ A \cap B \equiv B \cap A $ (Commutative)
  \item $ (A \cup B) \cup C \equiv A \cup (B \cup C) $ (Associative)
  \item $ (A \cap B) \cap C \equiv A \cap (B \cap C) $ (Associative)
  \item $ A \cup A \equiv A$ (Idompotent)
  \item $ A \cap A \equiv A$ (Idompotent)
  \item $ A \cup S \equiv S \cup A \equiv S $
  \item $ A \cap S \equiv S \cap A \equiv A $
  \item $ A \cup \phi \equiv \phi \cup A \equiv A $
  \item $ A \cap \phi \equiv \phi \cap A \equiv \phi $
  \item $ A \cup \overline{A} \equiv S$ (Definition of universal set)
  \item $ A \cap \overline{A} \equiv \phi$
  \item $ A \subseteq B \biimpl A \cup B \equiv B $
  \item $ A \subseteq B \biimpl A \cap B \equiv A $
  \item $ A \cap (B \cup C) \equiv (A \cap B) \cup (A \cap C) $ (Distributive)
  \item $ A \cup (B \cap C) \equiv (A \cup B) \cap (A \cup C) $ (Distributive)
\end{enumerate}
\par
{\color{magenta} \textbf{Significance}:
Some basic relations b/w sets so we can play with them.
\par}
\begin{proof}
Proof by using definitions of parent statements.
\end{proof}\par
\paragraph{4 parents} \hyperref[statement:Set]{Set}, \hyperref[statement:Universal Set]{Universal Set}, \hyperref[statement:Null Set]{Null Set}, \hyperref[statement:Identity]{Identity}, 
\paragraph{0 children} 



\addcontentsline{toc}{section}{Partition}
\begin{statement}[\textbf{Partition}]
\label{statement:Partition}\hspace*{0pt}\par
\end{statement}
\textbf{Description}:
A partition of a [\hyperref[statement:Set]{set}] S is a set $ \Lambda $ of [\hyperref[statement:Subset]{subset}]s of S such that
\begin{enumerate}[nolistsep]
  \item Intersection of any two elements of $ \Lambda \equiv \phi $ [\hyperref[statement:Null Set]{null set}]
  \item Union of all elements of $ \Lambda \equiv $ S
\end{enumerate}

\par
{\color{magenta} \textbf{Significance}:
A neat way to divide a set.
\par}
\begin{proof}
\proofbydefinition
\end{proof}\par
\paragraph{3 parents} \hyperref[statement:Set]{Set}, \hyperref[statement:Subset]{Subset}, \hyperref[statement:Null Set]{Null Set}, 
\paragraph{2 children} \hyperref[statement:Total Probability Theorem]{Total Probability Theorem}, \hyperref[statement:Bayes Theorem]{Bayes Theorem}, 


\end{document}
