\documentclass[../main.tex]{subfiles}

\begin{document}
\chapter{comparisions}

\defn{Equality}{
  Two things A and B are equal iff one can be replaced with another at any place in the context, written it as $ A = B $.
  If that does not hold they are called unequal, written as $ A \neq B $.
}{
  The definition itself is quite simple but I wanted to focus on the word \textbf{context}.
  For example, a Volkswagen car and a BMW car can be considered equal when the analysis is about classifying objects as cars and kittens.
  But they can be considered different if we are analyzing the mileages of different car brands.
}

\defn{Order}{
  Two unequal things are ordered iff there exists an notion that one thing comes before another. If A comes before B then we write $ A < B $ or $ B > A $.
}{
  There need not be order in all unequal pairs of things. For example there is no inherent order b/w a rabbit and a horse when listing out all animals. But there is an order when we consider the heights of two students when listing out heights of students in a class.
}

\end{document}

