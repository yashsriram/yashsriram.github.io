\documentclass[../main.tex]{subfiles}

\begin{document}
\chapter{Why are vector addition and scaling the way they are?}

\section{Why is the `RULE'  of vector addition the way it is?}

In case of 2D with vectors say $ \blockcomment{Column-Vector: x,y} \begin{pmatrix} x \\ y \end{pmatrix} $ and $ \blockcomment{Column-Vector: a,b} \begin{pmatrix} a \\ b \end{pmatrix} $ the addition rule for vectors is as follows
\[
  \begin{pmatrix} x \\ y \\  \end{pmatrix} + \begin{pmatrix} a \\ b \end{pmatrix} = \begin{pmatrix} x+a \\ y+b \end{pmatrix}.
\]
In a regular 2D co-ordinate frame with say $ A = \blockcomment{Column-Vector: 1,2} \begin{pmatrix} 1 \\ 2 \end{pmatrix} $ and $ B = \blockcomment{Column-Vector: -2,1} \begin{pmatrix} -2 \\ 1 \end{pmatrix} $ we get $ C = A + B = \blockcomment{Column-Vector: 1+(-2),1+1} \begin{pmatrix} 1+(-2) \\ 1+1 \end{pmatrix} = \blockcomment{Column-Vector: -1,3} \begin{pmatrix} -1 \\ 3 \end{pmatrix} $.

This can be extended to general N dimension vector.
\textbf{The point of all this is to re-enforce the notion that each of the numbers in the list (that creates the vector) is independent of all others and that each position behaves like a regular real number. This re-enforces the notion that vector is like a natural extension to the notion real numbers just that in this case multiple numbers go through the operation simultaneously.}

Otherwise there is no reason not to define the addition rule as something arbitrary like
\[
\blockcomment{Column-Vector: x,y} \begin{pmatrix} x \\ y \end{pmatrix} + \blockcomment{Column-Vector: a,b} \begin{pmatrix} a \\ b \end{pmatrix} = 
\blockcomment{Column-Vector: xa + yb^2, \frac{b - y}{x + a}} \begin{pmatrix} xa + yb^2 \\  \frac{b - y}{x + a} \end{pmatrix}
.\]

\section{Why is the `RULE' of scaling a vector the way it is?}
Same story. Here a real number is used to multiply each of the number in the list.
\[
2 * \blockcomment{Column-Vector: x, y} \begin{pmatrix} x \\  y \end{pmatrix} = \blockcomment{Column-Vector: 2x, 2y} \begin{pmatrix} 2x \\  2y \end{pmatrix}
.\]. The same thing can be extended to N dimensional vector. Here 2 is called a \textbf{SCALAR} as it scales each number in the list. Hence the word scalar came into existence instead of just being known as a number or something like `multiplying factor'. Mind == Blown!!
\end{document}

