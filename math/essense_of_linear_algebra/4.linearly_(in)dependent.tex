\documentclass[12pt]{article}
\usepackage[margin=1in]{geometry}
\usepackage{graphicx}
\usepackage{amsmath}
\usepackage{tikz}

\newcommand{\comment}[1]{}
\newcommand{\ihat}{\hat{\textbf{\i}}}
\newcommand{\jhat}{\hat{\textbf{\j}}}

\title{What does `linearly dependent' and `linearly independent' mean?}
\author{}

\begin{document}
\maketitle

\section{The title has the question}
A set of vectors are said to be \textbf{linearly dependent iff one of them can be expressed as a linear combination of others}. They are called linearly independent if they that cannot be done. Symbolically for a set of vectors $ v_1 $ $ v_2 $ ... $ v_n $ if we can find co-efficients $ \alpha_1 \alpha_2 ... \alpha_n $ such that \[ \alpha_1 * v_1  = \alpha_2 * v_2 + ... + \alpha_n * v_n \] then they are said to linearly dependent. If we cannot (as in there does not exist such set of co-efficients) then the set of vectors are called linearly independent.

Seems like a definition that is natural and will comes in handy.


\end{document}
