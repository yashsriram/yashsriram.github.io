\documentclass[../main.tex]{subfiles}

\begin{document}
\chapter{What is the deal with the terms `Span' and `Basis'?}

\section{Span}
\textbf{The span of any set of vectors (F) is the set of all vectors (S) that can be formed using linear combination of each of the vectors in (F)}. Symbolically, the span of vectors $ v_1 $ $ v_2 $ ... $ v_n $ (F) is set of all vectors formed by $ \alpha_1 * v_1 + \alpha_2 * v_2 + ... + \alpha_n * v_n $ (S) where $ \alpha_1, \alpha_2 ... \alpha_n $ are any real numbers.

The term span means something is similar to the normal meaning of the word span and the verb spanning itself i.e. like sort of to cover something.

\section{Basis}
\textbf{Basis of a set of vectors (S) is the set of linearly INdependent vectors (F) that span (S) i.e. the span of (F) is a superset or exactly equal to (S)}. Notice that
\begin{enumerate}
  \item By using the word linearly independent we are sort of putting a restriction on number of vectors in the set
\end{enumerate}

\section{But why these definitions?}
The idea is that from `space' or `spatial' point of view, span and basis are sort like inverse things. Given basis we can find the span, given a span we can find the basis. This sort of comes from the natural drive to generate and break down, I guess.

\end{document}

