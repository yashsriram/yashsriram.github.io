\documentclass[12pt]{article}
\usepackage[margin=1in]{geometry}
\usepackage{graphicx}
\usepackage{amsmath}
\usepackage{tikz}

\newcommand{\comment}[1]{}
\newcommand{\ihat}{\hat{\textbf{\i}}}
\newcommand{\jhat}{\hat{\textbf{\j}}}

\title{Relation b/w vector and point}
\author{}
\date{}

\begin{document}
\maketitle

\section{Why are vector and point so close?}
Apparently we have been taking for granted that a vector and corresponding point in appropriately dimensioned space is one and the same. But there is \textbf{a very implicit connection that often goes unnoticed. That is that when we convert a vector into a point we often multiply each number in the list by 1 and take the directions perpendicular to all others} . For example in 2D case $ \comment{Column-Vector: 3, -10} \begin{pmatrix} 3 \\  -10 \end{pmatrix} $ generally represents 3 units along $ \ihat $ and -10 units along $ \jhat $. \textbf{But that connection can be broken} .

The same vector $ \comment{Column-Vector: 3, -10} \begin{pmatrix} 3 \\  -10 \end{pmatrix} $ can represent 3 units along $ 2\ihat + -40\jhat $ and $ -300\ihat + 47.7\jhat $. These axes are not perpendicular, not unit length and can in generally even be parallel (linearly dependent). But still we can let the vector define lengths along these axes. Hell we can do whatever we want with math :P. But noticing this very implicit connection is enlightening.

\end{document}
