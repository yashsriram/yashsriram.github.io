\documentclass[../main.tex]{subfiles}

\begin{document}
\chapter{Vectors and space}

\section{A subtle assumption}
There is \textbf{a very implicit and subtle assumption that often goes unnoticed. That is that when we convert a vector into a point we often multiply each number in the list by 1 and take the directions perpendicular to all others} . For example in 2D case $ \blockcomment{Column-Vector: 3, -10} \begin{pmatrix} 3 \\  -10 \end{pmatrix} $ generally represents 3 units along $ \ihat $ and -10 units along $ \jhat $.

The same vector $ \blockcomment{Column-Vector: 3, -10} \begin{pmatrix} 3 \\  -10 \end{pmatrix} $ can represent 3 units along $ 2\ihat + -40\jhat $ and $ -300\ihat + 47.7\jhat $. These axes are not perpendicular, not unit length and can in generally even be parallel (linearly dependent).

But still we can let the vector define lengths along these axes. (Hell we can do whatever we want with math :P.) But actually letting go of the binding b/w numbers and so called `standard axes' can be useful in certain cases and noticing this very implicit assumption is just enlightening :).

\section{What if a co-ordinate space has no basis?}
\begin{theorem}
All co-ordinate spaces have at least one basis.
\end{theorem}

\begin{proof}
Let's try proof by contradiction.
Let's assume that there are some spaces which does not have a basis.
That means that there are no vectors that span that space.
Well if such a case exists \textbf{then that space has some points that cannot be expressed as linear combination of any of the vectors}.

But we already know that for any N dimensional co-ordinate space every point represents a vector and the relation b/w that vector and point is that by moving along each axis of co-ordinate frame by the corresponding number in the vector we arrive at the point by the definition of co-ordinate system.

If we take the vectors that define the co-ordinate frame as the set of our vectors then a linear combination of that vectors with numbers in our list shall produce the point.
\textbf{Therefore for any point in any co-ordinate space there exists a linear combination of vectors that produces the vector representing the point.}

Two conclusions derived from the hypothesis contradict each other, therefore the hypothesis is incorrect.

This means that for every co-ordinate space there must exist a set of vectors that span the space (most simplest of all are the vectors that define co-ordinate frame) (some of them can be removed until the rest are linearly independent).In other words \textbf{every space definitely has at least one basis}. It might be the case that the basis may span more than the required space, which is not really a problem according to our definition.

For example consider the vector space \{(1, 2, 3), (2, 1, 3), (0, 0, 0)\}. The set of vectors \{(1, 0, 0), (0, 1, 0) and (0, 0, 1)\} can be basis for the space, but their span is $ R^3 $.
\end{proof}

\section{Why do people say that $ \ihat $ and $ \jhat $ are `the' basis for 2D Cartesian Co-ordinate system ?}
Well it is not false that $ \ihat $ and $ \jhat $ are `one of the' basis for 2D co-ordinate system. But it is definitely not the only one i.e. not unique basis.
In fact there are many such basis sets for 2D co-ordinate system, infinitely many.

\begin{theorem}
Any 2D vectors that are linearly independent can be a candidate for the basis of the 2D co-ordinate space.
\end{theorem}

\begin{proof}
\end{proof}

\end{document}
