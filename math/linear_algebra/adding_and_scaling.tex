\documentclass[../main.tex]{subfiles}

\begin{document}

\section{vector addition}

\begin{definition}
  \label{def:vector_addition}

  \textbf{In case of 2D with vectors say $ \blockcomment{Column-Vector: x,y} \begin{pmatrix} x \\ y \end{pmatrix} $ and $ \blockcomment{Column-Vector: a,b} \begin{pmatrix} a \\ b \end{pmatrix} $ the addition rule for vectors is as follows
\[
  \begin{pmatrix} x \\ y \\  \end{pmatrix} + \begin{pmatrix} a \\ b \end{pmatrix} = \begin{pmatrix} x+a \\ y+b \end{pmatrix}.
\]
This can be extended to general N dimension vector.}

  \defusecase{This is to re-enforce the notion that each of the numbers in the list (that creates the vector) is independent of all others and that each position behaves like a regular real number. This re-enforces the notion that vector is like a natural extension to the notion real numbers just that in this case multiple numbers go through the operation simultaneously.
Otherwise there is no reason not to define the addition rule as something arbitrary like
\[
\blockcomment{Column-Vector: x,y} \begin{pmatrix} x \\ y \end{pmatrix} + \blockcomment{Column-Vector: a,b} \begin{pmatrix} a \\ b \end{pmatrix} = 
\blockcomment{Column-Vector: xa + yb^2, \frac{b - y}{x + a}} \begin{pmatrix} xa + yb^2 \\  \frac{b - y}{x + a} \end{pmatrix}
.\]
}
\end{definition}


\section{vector scaling}

\begin{definition}
  \label{def:vector_scaling}

  \textbf{For a real number `s' the scaling rule for vectors is as follows \[
s * \blockcomment{Column-Vector: x, y} \begin{pmatrix} x \\  y \end{pmatrix} = \blockcomment{Column-Vector: sx, sy} \begin{pmatrix} sx \\  sy \end{pmatrix}
.\]. The same thing can be extended to N dimensional vector.}

  \defusecase{Same story. Here `a'  is called a \textbf{scalar} as it scales each number in the list. Hence the word scalar came into existence instead of just being known as a real-number or something like `multiplying factor'. Mind == Blown!!}

\end{definition}

\pagebreak
\end{document}
