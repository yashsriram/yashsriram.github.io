\documentclass[../main.tex]{subfiles}

\begin{document}
\section{what the hell is a vector?}

Well, as the case with any definition we can always choose to make it whatever we want. :P
But in a less comical note many problems in the (real) world apparently can be represented and solved by using list of numbers as a fundamental tool.

\begin{definition}
  \label{def:vector}

  \textbf{The vector as just a list of numbers, in which the order of numbers matter.}

  \defusecase{Provides the abstraction of a `super number'  i.e. operations on it can be seen as operations to all number simultaneously}
\end{definition}


\section{dimension}
\begin{definition}
  \label{def:vector_dimension}

  \textbf{The dimension of a vector [\ref{def:vector}] is just the count of numbers in the list.
A vector with dimension N can be called ND-vector (spoken as N dimensional vector).}

  \defusecase{A natural property of vector}
\end{definition}

For each of the following vectors its dimensionality is written in the third column.

\begin{table}[ht]
  \centering
  \begin{tabular}{ c  c  c }
    Vector name & Value & Dimension\\
    $ V_0 $ & () & 0\\
    $ V_1 $ & (1) & 1\\
    $ V_2 $ & ($\sqrt{2}$) & 1\\
    $ V_3 $ & (-100, $\sqrt{3}$) & 2\\
    $ V_4 $ & (0, 0.1) & 2\\
    $ V_5 $ & (0, 0, 0) & 3\\
    $ V_6 $ & (0, 1, 2, 3) & 4\\
  \end{tabular}
\caption{Dimensions of Vectors}
\label{tab:dim}
\end{table}

\pagebreak
\end{document}

