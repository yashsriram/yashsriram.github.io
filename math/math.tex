\documentclass[12pt]{article}
\usepackage[margin=1in]{geometry}
\usepackage{graphicx}
\usepackage{amsmath}
\usepackage{tikz}
\usepackage{hyperref}

\newcommand{\comment}[1]{}
\newcommand{\ihat}{\hat{\textbf{\i}}}
\newcommand{\jhat}{\hat{\textbf{\j}}}

\title{What is Mathematics?}
\author{}

\begin{document}
\maketitle

According to Wikipedia and me there is no single consensual definition.
But according to me, the definition would be \textbf{the most systematic study of things humanly possible}.
The word `humanly' is important, as even Mathematics is after all a human made thing and nothing soooo `absolute' about it.
The word `humanly' inherently makes the process of anything related to that (Even mathematics) an iterative process.
In that nothing we conclude, discover, invent ever can be trusted a full 100\% and should constantly be challenged.
It is in the balance between having trust and challenging it simultaneously lies any `conceivable progress'.
If the balance tips left it is called `arrogance', else if it tips right it is called  `insanity'.

\end{document}

