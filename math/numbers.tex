\documentclass[../main.tex]{subfiles}

\begin{document}
\chapter{numbers}

\defn{Number}{
  A number is a notion to count or to label or measure things.
}{
  Some uses of numbers are
  \begin{enumerate}[noitemsep,topsep=0pt]
    \item Count number of apples in a bag - There are 37 apples in a bag
    \item Measure the length of a pencil - The length of the pencil is 3.1468 cm
    \item Label each participant in a marathon - The participant number 10714 won the marathon
  \end{enumerate}
}

\defn{Zero}{
  The notion of absence.
  Symbolically written as 0.
  Can be considered as a number.
}{
  For example in case of counting it can represent the case when there is no balls left in an urn.
}

\defn{NaturalNumbers}{
  The OrderedSet of all numbers ( Number ) used for counting and ordering is called the set of natural numbers. It is denoted by $ \mathbb{N} $.
}{
  Encompasses all possible counts.
}

\defn{WholeNumbers}{
  The Union of NaturalNumbers and Set which contains only Zero . It is denoted by $ \mathbb{W} $.
}{
  The name whole is given because we are gonna define numbers that will represent non-whole entities.
}

\defn{Addition}{
  Addition is a rule to combine numbers.
  Each sets of numbers there are 
  The union of natural numbers and set which contains only zero. It is denoted by $ \mathbb{W} $.
}{
  The name whole is given because we are gonna define numbers that will represent non-whole entities.
}

\end{document}

