\documentclass[./main.tex]{subfiles}

\begin{document}
\chapter*{preface}
This is not a re-write of any book.
This is the written form of my perspective and understanding of mathematics.
In big picture this text is a directed acyclic graph (DAG) of statements.
This text has the following constructs
\begin{enumerate}
  \item \textbf{Definition}: A statement that is starting point in the DAG.
  \item \textbf{Theorem}: A statement that is intermediate point in the DAG.
  \item \textbf{Proof}: A logical flow that connects statement A to B, such that if A is true only then B is true else B is false.
\end{enumerate}

If there are circles in the statement graph, then there is a circular dependency among statements for them to be true. This makes not much sense to me and therefore DAG.

Definitions and theorems \textbf{always} have a reason to be considered i.e. there is a reasonable answer to the question `why the hell are we even spending time on this?' I shall try to write this for every definition.
\end{document}

