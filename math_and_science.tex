\documentclass[12pt]{article}
\usepackage[margin=1in]{geometry}
\usepackage{graphicx}
\usepackage{amsmath}
\usepackage{tikz}
\usepackage{hyperref}

\newcommand{\comment}[1]{}
\newcommand{\ihat}{\hat{\textbf{\i}}}
\newcommand{\jhat}{\hat{\textbf{\j}}}

\title{What is Mathematics (and probably science too)?}
\author{}

\begin{document}
\maketitle

According to Wikipedia and me there is no single consensual definition.
But according to me, the definition would be \textbf{the most systematic study of things humanly possible}.
The word `humanly' is important, as even Mathematics is after all a human made thing and nothing soooo `absolute' about it.
The word `humanly' inherently makes the process of anything related to that (Even mathematics) an iterative process.
In that nothing we conclude, discover, invent ever can be trusted a full 100\% and should constantly be challenged.
It is in the balance between having trust and challenging it simultaneously lies any `conceivable progress'.
If the balance tips left it is called `arrogance', else if it tips right it is called  `insanity'.
If the balance is right I feel that it produces a sense of happiness, fun and progress.
Now I believe that this feeling is what human beings pursuing science crave for.
Me personally have a application-oriented mindset, as in if I can apply my knowledge to create/improve a tangible system, it produces a great sense of joy to me.
So I guess if I bring some kind of application to every piece of new thing I learn then I might have a deeper (a more intimate) insight in it.

So the key points are
\begin{enumerate}
  \item It is not only good but also important to challenge what you trust with control.
  \item Periodically it is very good to think that whatever you learned until now is completely wrong and non-sense and reset your brain and knowledge fully.
  \item The whole point of pursuing math and science it to pursue the joy in it, NEVER FORGET THIS. So better stop feeling so seriously about it.
  \item Draw an application to everything you learn, you will learn better this way.
\end{enumerate}
\end{document}

