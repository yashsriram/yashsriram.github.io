%% $Id: multido-doc.tex 317 2010-05-14 21:54:15Z herbert $
\documentclass[12pt]{article}
\usepackage{fp,multido}
\let\MDfv\fileversion
\usepackage{url}

\makeatletter

%% PAGE PARAMETERS:
\setlength{\parindent}{0pt}          % paragraph indent
\setlength{\parskip}{4pt plus 1pt minus 1pt}
\setlength{\topmargin}{-2.5pc}
\setlength{\headheight}{12pt}        % height of running head
\setlength{\headsep}{30pt}           % distance between header and text
\setlength{\textheight}{54pc}        % height of text on page
%%
%% DATES, VERSIONS AND TITLES:
%%
\def\expanddate#1/#2/#3/{\year=#1 \month=#2 \day=#3}
\begingroup
  \expandafter\expanddate\filedate/
  \xdef\thefiledate{\today}
\endgroup
\date{Version \fileversion\\ \thefiledate}
\def\@maketitle{%
 \begin{center}
   {\Large\bf \@title \par}
   \vskip 1.2em {\lineskip .5em
   \begin{tabular}[t]{c}\@author\end{tabular}\par}
   \vskip .8em {\@date}%
  \end{center}
  \par
  \vskip .5em}
%%
%% PAGE STYLE:
%%
\pagestyle{myheadings}
\markright{Documentation for multido.tex \hfill
  v.\fileversion\hskip 1em \thefiledate\hfill}%
%%
%% VERBATIM:
%%
\begingroup
  \catcode`\{=12 \catcode`\}=12
  \catcode`\(=1 \catcode`\)=2
  \catcode`\+=0 \catcode`\\=12
  +gdef+|(\)          % \| = character \
  +gdef+{({)          % \{ = character {
  +gdef+}(})          % \} = character }
+endgroup
\def\Backslash{\protect\|}
%%
%% Define some functional font commands:
%%
\def\MainFont{\tt}      % For macro definitions.
\def\UsageFont{\tt}           % For in-line macro names.
\def\InlineFont{\tt}          % For other in-line snipets of code.
\def\MetaFont{\rm\it}            % For meta arguments.
%%
%% Short-cuts for font commands:
%%
\def\s#1{\string#1}
\def\t#1{{\InlineFont\string#1}}   % For other in-line code.
\def\m#1{{\MetaFont #1\/}}         % For just putting things in italics
\def\M#1{{\InlineFont\{\m{#1}\}}}  % Arg is in italic, enclosed in tt braces.
\let\N\t                           % For in-line macro names.
\catcode`\<=13 \def<#1>{\m{#1}}    % <meta>
%%
%% This is for listing macro definitions in a quote-like environment.
%%
\begingroup
  \catcode`\[=1 \catcode`\]=2
  \catcode`\{=13 \catcode`\}=12
  \gdef\@MD[%
    \catcode`\{=13 \catcode`\}=12
    \def{##1}[\{[\MetaFont ##1]\}]]
\endgroup
\def\MD{%
  \quote
  \begingroup
    \@MD
    \def\end{\endgroup\end}%
    \def\\{\@centercr\s}%
    \MainFont
    \s}
\def\endMD{\endquote}
%%
%% END PREAMBLE:
%%
\catcode`\"=13
\def"{\verb"}
\catcode`\@=12
\begin{document}
%%
%%%%%%%%%%%%%%%%%%%%%%%%%%%%%%%%%%%%%%%%%%%%%%%%%%%%%%%%%%%%%%%%%%%%%%%%%%%%%

\title{Documentation for multido.tex, version \MDfv:\\[2pt]
  A loop macro for Generic \protect\TeX}
\author{Timothy Van Zandt%
  \thanks{Author's address: Department of Economics, Princeton University,
  Princeton, NJ 08544-1021, USA. Internet: \protect\url{tvz@Princeton.EDU}. 
  This document was edited by Rolf Niepraschk \protect\url{Rolf.Niepraschk@gmx.de} and
  Herbert Voss \protect\url{hvoss@tug.org} to get it run with LaTeX2e (2006-01-01)}}
\maketitle

"multido.tex"/"multido.sty" contains the \N\multido\ macro, which is a loop
facility for Generic TeX. This macro happens to be useful for drawing
pictures, and was originally developed for the PSTricks
package,\footnote{PSTricks is an extensive collection of PostScript-based
macros for Generic TeX. It is available from \url{CTAN://graphics/pstricks/}.} 
but you can use it for other purposes as well.

A special feature is support of fixed-point addition. For example, PSTricks
uses the \N\multido\ to put numbers on axes, much like in the following
\LaTeX\ example:
\begin{verbatim}
    \setlength{\unitlength}{1cm}
    \small
    \begin{picture}(8,1)(0,-.5)
      \put(0,0){\vector(1,0){8}}
      \multido{\i=0+1,\n=0+0.25}{8}{%
        \put(\i,-.1){\line(0,1){.2}}
        \put(\i,-.2){\makebox(0,0)[t]{\n}}}
    \end{picture}
\end{verbatim}
\begin{center}
    \setlength{\unitlength}{1cm}
    \small
    \begin{picture}(8,1)(0,-.5)
      \put(0,0){\vector(1,0){8}}
      \multido{\i=0+1,\n=0+0.25}{8}{%
        \put(\i,-.1){\line(0,1){.2}}
        \put(\i,-.2){\makebox(0,0)[t]{\n}}}
    \end{picture}
\end{center}

The general syntax for \N\multido\ is:
\begin{MD}
  \multido{variables}{repetitions}{stuff}
\end{MD}
<stuff> is whatever you want repeated; it can be any balanced \TeX\ input.
<repetitions> is the number times <stuff> is repeated.

The first argument is the interesting one. <variables> is a comma-separated
list of variable declarations.\footnote{Don't use commas to mark the decimal
point within the \protect<variables> argument, as they will be confused for
delimiters.} Each variable declaration is of the form:
\begin{center}\tt
  <variable> = <initial value> + <increment>
\end{center}
<variable> is a command sequence that can be used in <stuff>. It is initially
set to <initial value>, and is then incremented by <increment> with each
repetition.

The first letter of the variable name determines the variable type. There are
four variable types:
\begin{description}
\item[Dimension (d or D)] The initial value and the increment should be
dimensions (lengths, in \LaTeX\ parlance). The substitution text is a
dimension, with "sp" units. E.g., "\dx=4cm+5pt".\footnote{For PSTricks users,
the unit is optional.}

\item[Number (n or N)] The initial value and increment should be integers or
numbers with the same number of digits to the right of the decimal. The one
exception is that it is always OK for the initial value to be an integer.
There can be at most 8 digits on each side of the decimal. The substitution
text is a number, with fixed-point addition. E.g., "\n=3+7.05",
"\Nx=5.30+-1.25".

\item[Integer (i or I)] The initial value and increment should be integers.
This gives the same result as using a number variable, but it is faster. E.g.,
"\I=2+-1".

\item[Real (r or R)] The initial value and increment should be integers or
numbers with at most 4 digits on each side of the decimal. The substitution
text is a number, but with floating point addition and occasional small
errors. This gives a less satisfactory result than using a number variable,
but it is faster. E.g., "\ry=4.2+1.05".
\end{description}

Here are some examples that illustrate how the substitution text is
determined:
\begin{quote}
  "\multido{}{10}{\TeX\ }"\\[3pt]
    \hbox to 2em{}\multido{}{10}{\TeX\ }{}\\[8pt]
  "\multido{\d=2pt+3pt}{5}{\d, }"\\[3pt]
    \hbox to 2em{}\multido{\d=2pt+3pt}{5}{\d, }{}\\[8pt]
  "\multido{\n=2+3}{10}{\n, }"\\[3pt]
    \hbox to 2em{}\multido{\n=2+3}{10}{\n, }{}\\[8pt]
  "\multido{\i=2+-3}{10}{\i, }"\\[3pt]
    \hbox to 2em{}\multido{\i=2+-3}{10}{\i, }{}\\[8pt]
  "\multido{\r=2+3.05}{6}{\r, }"\\[3pt]
    \hbox to 2em{}\multido{\r=2+3.05}{6}{\r, }{}\\[8pt]
  "\multido{\n=2.00+-3.05}{8}{\n, }"\\[3pt]
    \hbox to 2em{}\multido{\n=2.00+-3.05}{8}{\n, }{}
\end{quote}

Here are some details about the choice of names:
\begin{itemize}
\item Your computer won't explode if you use names that conflict with \TeX\
internal commands, but you might want to check name conflicts if you get
inexplicable errors. The command \N\MultidoCheckNames\ can be useful in this
case. It causes \N\multido\ to report an error whenever you use a variable
name that is already defined. But see the next item.
\item The whole \N\multido\ loop is grouped. This means, e.g., that although
"\i" is a Plain \TeX\ command sequence (giving a dotless ``\i''), you can use
the variable "\i" if you do not use any dotless i's in <stuff> (and if you do
not use \N\MultidoCheckNames).
\end{itemize}

Here are a few more details:
\begin{itemize}
\item \N\Multido\ commands can be nested.
\item Spaces after a \N\multido\ command are ignored. This makes \N\multido\
more hospitable for pictures.
\item Spaces between the various parts of the <variables> argument are
ignored.
\end{itemize}

And finally here a few special features, some of which are of interest mainly
macro writers and other \TeX nicians:
\begin{itemize}
\item The material that is repeated is not grouped, so that you can insert
your own recursive routines.
\item There is a variant, "\mmultido", which works just like "\multido" except
that the variables are all incremented once before starting.
\item If you use "\Multido" or "\MMultido" instead of "\multido" or
"\mmultido", resp., then the whole loop is not grouped. This can be useful,
e.g., for making entries in an alignment environment. However, these cannot be
nested within any "\multido" macro.
\item If the number of repetitions is a negative number, the variables are
incremented backwards.
\item The count register "\multidocount" keeps track of the number of current
iteration.
\item The command "\multidostop" causes the "\multido" loop to quit at the end
of the current iteration.
\item Fixed point addition is performed by \N\fpAdd\ and \N\fpSub:
\begin{MD}
  \fpAdd{num1}{num2}{command}\\
  \fpSub{num1}{num2}{command}
\end{MD}
<num2> is added to or subtracted from <num1>, and the answers is stored in the
command sequence given as the third argument. The rules about decimals and so
on that apply to number variables apply here as well. E.g., after
\begin{verbatim}
    \fpSub{1.75}{-0.15}{\answer}
\end{verbatim}
\fpSub{1.75}{-0.15}{\answer}
the definition of "\answer" is \answer.
\end{itemize}

\subsection*{Changes:}

\begin{description}
\item[V1.1] Fixed bug in \N\FPadd\ that gave wrong answer for, e.g., $3.4 +
-0.2.$

\item[V1.2] Made unit optional for dimension variables when using PSTricks.

\item[V1.3] Now 0 repetitions really gets 0 repetitions. "\def\multido@"
changed to "\long\def\multido@".

% RN modification begin
%\end{description}
% RN modification end

\item[V1.4] Small change to make it compatible with PSTricks v0.93 and later.

\item[V1.41] Fix bug when using the fp-package (same macronames) and small
changes to this documentation (RN/HV)
\end{description}

\end{document}
%% END multido-doc
