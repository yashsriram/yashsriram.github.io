\documentclass[12pt]{article}
\usepackage[margin=1in]{geometry}
\usepackage{graphicx}
\usepackage{amsmath}
\usepackage{tikz}
\usepackage{hyperref}

\newcommand{\comment}[1]{}
\newcommand{\ihat}{\hat{\textbf{\i}}}
\newcommand{\jhat}{\hat{\textbf{\j}}}

\title{The purpose of this Journal}
\author{}
\date{}

\begin{document}
\maketitle


Well I think (with some rigor) about a lot of things and it seems that after a point I am not able to keep track of them.
I forget some important conclusions, perspectives, schools of thought, jewels of wisdom and such important things that are fruits of a lot of effort, suffering and pain in various aspects.
And I have to re-start my train of thought on them, which is tedious and dis-motivating after a point of time.

\paragraph{What it is for?}
Therefore I guess the main purposes of this journal are to
\begin{enumerate}
  \item Write down stuff so that
    \begin{enumerate}
      \item I don't lose it permanently from memory.
      \item I don't have to remember the stuff I don't always need to.
      \item I use my brain as more of a compute machine than a data storage machine.
    \end{enumerate}
  \item Search using a computer from this stored knowledge of thought and wisdom.
  \item Build on my previous knowledge, update and renew that instead of always starting from scratch.
\end{enumerate}

\paragraph{What it is NOT for?}
This journal is not to
\begin{enumerate}
  \item Encapsulate all of my knowledge only the highly condensed version of it, just the diamonds that I find in the mines of my thoughts.
\end{enumerate}

\end{document}
