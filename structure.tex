\documentclass[./main.tex]{subfiles}

\begin{document}
\section{structure}
This is not a re-write of any book.
This is the written form of my perspective and understanding of mathematics.

In big picture this text is supposed to be a directed acyclic graph (DAG) of statements.
If there are circles in the statement graph, then there is a circular dependency among statements for them to be true. This makes not much sense to me and therefore DAG.

The nodes of the DAG (statements) can be any one of the following
\begin{enumerate}
  \item \textbf{Axiom}: these nodes should not have any parents
  \item \textbf{Definition}: these nodes can have definitions as parents
  \item \textbf{Theorem}: can have axioms, definitions and theorems as parents
\end{enumerate}

An edge in the DAG is \textbf{Proof} that starts from statement node A to statement node B representing that
\begin{verbatim}
  if (all parents of statement A are true) {
    A is true
  } else {
    A is false
  }
\end{verbatim}

The purpose of DAG is that for a given a statement, it should be able to provide all the statements, theorems (called lemmas in this context), definitions and axioms that are required to prove it (true). The hope is that by seeing the bigger picture we shall have better insights and find new things.

Strictly speaking I don't see much difference b/w an axiom and a definition.
But just for the sake of sanity I am gonna use those, as anyway it doesn't make anything incorrect.

Definitions and theorems \textbf{always have a significance} i.e. there is a reasonable answer to the question `why the hell are we even spending time on this?' I shall try to write this for every definition.

\pagebreak
\end{document}

